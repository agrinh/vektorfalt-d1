\documentclass[12pt,a4paper]{article}
\pdfoutput=1

\usepackage[utf8]{inputenc}
\usepackage[T1]{fontenc}
\usepackage[swedish]{babel}
\usepackage{amsmath}
\usepackage{lmodern}
\usepackage{units}
\usepackage{icomma}
\usepackage{color}
\usepackage{graphicx}
\usepackage{bbm}
\newcommand{\N}{\ensuremath{\mathbbm{N}}}
\newcommand{\Z}{\ensuremath{\mathbbm{Z}}}
\newcommand{\Q}{\ensuremath{\mathbbm{Q}}}
\newcommand{\R}{\ensuremath{\mathbbm{R}}}
\newcommand{\C}{\ensuremath{\mathbbm{C}}}
\newcommand{\rd}{\ensuremath{\mathrm{d}}}
\newcommand{\id}{\ensuremath{\,\rd}}
\usepackage{hyperref}
\usepackage{subfig}
\setlength\parindent{0pt}

\begin{document}

\title{Vektorfält Datorinlämning 1}
\author{Agrin Hilmkil -- agrin@student}
\date{September 2014}
\maketitle

\begin{abstract}
I denna rapport presenteras lösningar för datoruppgift 1 i FFM232 Vektorfält och klassisk fysik. Samtliga lösningar och rapport finns tillgängliga på https://github.com/agrinh/vektorfalt-d1
\end{abstract}

\newpage
\tableofcontents
\newpage

\section{Visualisering av ett skalärt fält}

Här visualiseras gravitations potentialen som beskriver två stjärnor med position $r_1$ och $r_2$ relativt masscentrum och med massor $m_1$ respektive $m_2$:
\begin{equation}
\phi (\vec{r}) = - \frac{Gm_1}{|\vec{r} - \vec{r_1}|} - \frac{Gm_2}{|\vec{r} - \vec{r_2}|} - \frac{1}{2}(\vec{\omega} \times \vec{r})^2
\end{equation}

Där $\omega$ ges av:
\begin{equation}
\vec{\omega} = \sqrt{\frac{G(m_1 + m_2)}{|\vec{r_1}-\vec{r_2}|}}\ \ \hat{z}
\end{equation}

Vidare används följande värden:
\begin{table}[h!]
\centering
\begin{tabular}{c|c|c}
 Symbol & Beskrivning & Värde \\
 \hline
 $G$ & Newtons gravitationskonstant & $6.673\cdot10^{-11}$ \unit{N$\cdot(\frac{m}{kg})^2$} \\
 $m_s$ & 1 Solmassa & $1.989\cdot10^{30}$ \unit{kg} \\
 $d$ & Avstånd mellan stjärnorna & $7\cdot10^{5}$ \unit{km}
\end{tabular}
\caption{Värden}
\label{tab:konstanter}
\end{table}

Slutligen kan vi enkelt bestämma $r_1$ och $r_2$ relativt masscentrum som sätts till origo genom att placera dessa längs x-axeln, låta den större stjärnan ligga åt negativa x och lösa:
\begin{equation}
\begin{array}{lcl}
x_1 \cdot m_1 + x_2 \cdot m_2 = 0 \\
|x_1| + |x_2| = d
\end{array}
\end{equation}

Där $x_i$ är x komponenten av $r_i$. Vi får då:
\begin{equation}
\begin{array}{lcl}
\vec{r_1} = - \frac{d}{1 + \frac{m_1}{m_2}} \ \hat{x} = -2\cdot10^{5} \ \hat{x} \\
\vec{r_2} = \frac{m_1}{m_2} |x_1| \ \hat{x} = 5\cdot10^{5} \ \hat{x}
\end{array}
\end{equation}

\subsection{Uppgift A}
Visualiseringen syns i figur \ref{fig:stars}. Då stjärnorna sammanfaller med ekvipotentialytorna ser vi att de för små radier är cirkulära medan de för större radier blir desto mer äggformade mot varandra innan de flyter ihop.
\begin{figure}[!ht]
\centering
\includegraphics[width=25em]{full_stars.eps}
\caption{\label{fig:stars} Nivåytor av gravitationspotentialen}
\end{figure}

\subsection{Uppgift B}
Som förväntat ser vi i figur \ref{fig:stars_grad} på gradienten utritad med \texttt{quiver} att en testpartikel med positiv massa dras in mot stjärnorna.
\begin{figure}[!ht]
\centering
\includegraphics[width=25em]{full_stars_arrows.eps}
\caption{\label{fig:stars_grad} Nivåytor av gravitationspotentialen med gradienten utritad}
\end{figure}


\section{Visualisering av ett vektorfält}
De generarade fältlinjerna för denna uppgift syns i figur \ref{fig:fluid}, där figur a och b är för respektive deluppgift. I båda fallen placeras startpunkter på samma ställen (utom en punkt som flyttades i uppgift a för att undvika överlapp med en annan fältlinje). De konstanter som användes är angivna i tabell \ref{tab:fluid_konstanter}.\\



Ekvationerna som används är de i uppgiften angivna hastighetsfälten. Dessa löses med \texttt{ode15s} från varje startpunkt och plottas därefter i figuren med olika färger. Vidare ritas även här vektorfältet ut med hjälp av \texttt{quiver} för att tydligaregöra virvlarna.

\begin{table}[h!]
\centering
\begin{tabular}{c|c}
 Symbol & Värde \\
 \hline
 $J$ & $2$ \\
 $x_0$ & $10$
\end{tabular}
\caption{Konstanter för hastighetsfältet}
\label{tab:fluid_konstanter}
\end{table}


\begin{figure}[!ht]
    \centering
    \makebox[\textwidth][c]{
        \subfloat[]{\includegraphics[width=25em]{fluid_motion_a.eps}}
        \subfloat[]{\includegraphics[width=25em]{fluid_motion_b.eps}}
    }
    \caption{\label{fig:fluid} Fältlinjer i fluid med virvlar. Cirkel markerar startposition för en testpartikel och X dess slutposition.}
\end{figure}

\end{document}